
% Document settings
\documentclass[11pt]{article}
  \usepackage[margin=.7 in]{geometry}
\usepackage[pdftex]{graphicx}
\usepackage{palatino}
\usepackage{multirow}
\usepackage{setspace}
\pagestyle{plain}
\usepackage{hyperref}
\usepackage{color}
\usepackage{verbatim}
%


\onehalfspacing

\setlength{\parindent}{2em}
\setlength{\parskip}{0em}
%\renewcommand{\baselinestretch}{2.0}

\begin{document}

 \noindent\LARGE Math 324 E, Spring 2020 

 \medskip

 \noindent\LARGE Advanced Multivariable Calculus \\

 \medskip

 \noindent\large MWF 8:30-9:20 am	on Zoom. \\
 See \href{https://sites.math.washington.edu/~neptamin/324Sp20/Math%20324%20E%20Spring%202020.html}{Class website} for meeting ID.\\

\vspace{2mm}


\noindent\large Instructor: Nikolas Eptaminitakis\\
\noindent\large Email: \href{mailto:neptamin@uw.edu}{neptamin@uw.edu} \\
%\large Course website:\url{} \\
\noindent\large Office Hours: See \href{https://sites.math.washington.edu/~neptamin/324Sp20/Math%20324%20E%20Spring%202020.html}{Class website}

%\begin{center} A statement about the changeable nature of this syllabus. \\
%\end{center}

% Course details
\section*{Course description and prerequisites}
 The topics we will cover include Double and Triple Integrals, the Chain Rule, Directional Derivatives, Vector fields, Line and Surface integrals, the Grad, Curl and Div operators, the Divergence Theorem, and Green's and Stokes' Theorems. This course is a continuation of the 120 level calculus courses, and its content partly overlaps with material covered in Math 126. 
 % \textbf{Warning: This class becomes challenging towards the end, and the last few sections assume a good understanding of the entire material covered during the quarter}. Especially if you took Math 126 recently, you might find the first couple of weeks easy to follow, but this might not be the case later.

As for prerequisites, I will assume you are comfortable with basic differentiation and integration in one variable and basic algebraic manipulations. However, you shouldn't expect to have to deal with some really complicated integration tricks and I will be reviewing past material you might not remember whenever necessary.
%\textbf {Prerequisite(s):} None.

%\textbf {Note(s):} A minimum grade of C is required in this course to progress to COURSE.

%\textbf {Credit Hours:} 3 \\

\section*{Lectures and Office Hours}

At least initially, the lectures will be delivered real time on-line through Zoom at the regular time the class meets.
You will have the opportunity to stop me and ask a question.
The lectures will also be recorded and uploaded to Canvas. 
To the best of my knowledge, Zoom does not offer an option to not record all participants' voices, so if you ask a question your voice \textit{will} be recorded. 
However, I will be removing students' voices from the version of the lecture posted on Canvas.
% Due to space restrictions, older videos might have to be deleted at some point so if you think you will need them you should feel free to download them.
In case real time lecturing ends up not being practical I will resort to prerecorded lectures posted to Canvas and inform you about it.

\medskip

\noindent Office hours are also going to be held remotely on Zoom. Those will not be recorded.

\section*{Communication}

You can reach me through email or Canvas.
You should generally be expecting a response within 24 hours.
\textbf{If for whatever reason you do not have access to your UW email please let me know immediately through Canvas.}


\section*{Textbook, Lecture Notes and other Study Resources}


I will generally be following the \textbf{textbook} \emph{``Multivariable Calculus-(Early Transcendentals)''}, 8th Edition by James Stewart. We will cover part of chapter 14 and all of Chapters 15 and 16. The textbook can be found on websassign (so you don't need to purchase a physical copy separately if you don't want to).
 On the class website you can also find my \textbf{lecture notes},
and the recorded lectures on Canvas.
%Also, on the website you will find some excellent \textbf{review sheets} prepared by Dr. Andrew Loveless that provide a good summary of the material.

\medskip

\noindent There is a Piazza discussion forum where you are encouraged to post questions and form study groups:
\begin{center}
  \url{https://piazza.com/class/k86un8mpeay7ks}.
\end{center}

\medskip

\noindent Another resource you might want to consider is CLUE, which will be operating remotely:
\begin{center}
 \url{http://depts.washington.edu/aspuw/clue/home/}. 
\end{center}
 
 

\section*{Homework}

There will be homework assigned roughly once a week, and it will be submitted and graded through \href{https://www.webassign.net/washington/login.html}{\textbf{WebAssign}}. You will need a WebAssign access code, but if you have already purchased a Lifetime of the Edition access to WebAssign for the 8th edition of Calculus: Early Transcendentals of Stewart in Math 124-6, it should still work form Math 324. \textbf{I will not give individual extensions to homework assignments}. To make up for that, at the end I will add 10\% to your homework average score, up to 100\% (that is, 87\% will count as 97\% and 92\% will count as 100\%).


\section*{Exams}

There will be 2 midterms and a final at the following dates:
\begin{center}
\begin{tabular}{l l l}
  Midterm 1 & Wednesday, April 22& 8:30 - 9:20 am\\
  Midterm 2 & Wednesday, May 13& 8:30 - 9:20 am\\
  Final Exam & Tuesday, June 9& 	8:30 - 10:20 am
\end{tabular}
\end{center}
\begin{itemize}
  \item Exams will be distributed online. You can start each midterm within 1.5 hour before or after the normal starting time. Then you have 1 hour to submit your work. That includes 10 minutes allowed for uploading your work.
  For the final, you will be able to start within 2.5 hours before or after the normal starting time.
The time allocated for uploading your work will be 15 minutes.
  \item The exams are open book/internet. \textbf{However, you are strictly prohibited from collaborating, asking others in person or oniline how to do the problems or for hints, or sharing the exam questions with other people, whether they are students in the class or not. The exam reflects your own work and understanding of the material.}



Remember that you are expected to abide by the University's \href{https://www.washington.edu/cssc/for-students/student-code-of-conduct/}{{\color{black}\underline{Code of Conduct}}} and the 
\href{https://depts.washington.edu/grading/pdf/AcademicResponsibility.pdf}{{\color{black}\underline{Statement of Academic Responsibility}}}.
Academic misconduct will not be tolerated.






  \item If you have a compelling, documented, and legitimate reason to miss an exam please inform the instructor as soon as possible.
\end{itemize}








% No make-up midterms will be given. If you have a documented and unavoidable reason to miss a midterm, I will give you a grade based on the median of the exam and your performance in the other two exams. Namely, for each of the two other exams I will calculate how many standard deviations away from the median you scored and average the two numbers. Then, if the resulting number is $x$, your grade for the exam you missed will be $x$ standard deviations away from the median of this exam. This formula will give a more fair result in case the level of difficulty in the various exams is not the same. \textbf{In any case, to be excused for an exam you need to inform me in advance.}

\section*{Quizzes}
Quizzes will typically be short exercises that shouldn't take you more than 10-15 minutes to complete. 
They will be graded on completion and they will be available online for 24 hours once a week, starting on week 2.
The worst quiz will be dropped.


\begin{comment}
  
  section*{Worksheets}

  There will be a number of worksheet sessions done in class. You are strongly encouraged to work on them in groups and you can turn in your solutions, which will not be
  graded, but I will look at them so I can point out common mistakes and misconceptions before
  the actual exams.Their purpose is to encourage you to discuss some problems
  with other people in class, ask questions, and write down a
  solution for some problems in a stress-free
  environment. You will be notified about worksheet sessions ahead of time.
\end{comment}



%  
%  \section*{Mathematica tutorial}


% I will run a tutorial on Wolfram Mathematica, on a date and time to be announced soon. Mathematica is a powerful symbolic computation program that can be used for computation of integrals both symbolically and numerically, differentiation, visualization of curves and surfaces etc. We will see how to use some commands that are relevant to the course. Participation to the tutorial is completely optional, it does not count towards your grade, and you will not be tested on anything related to it. Mathematica is provided for free by UW College of Engineering to UW students, and you can find it under this link: \url{https://www.engr.washington.edu/mycoe/computing/software/install_mathematica}. Feel free to let me know if you have trouble installing it.




\section*{Grade Distribution}
 \indent The grade distribution is as follows\\
\hspace*{40mm}
\begin{center}
\begin{tabular}{ l l }
Homework & 12\% \\
Quizzes & 3\%\\
Midterm  1& 25\% \\
Midterm 2  & 25\% \\
Final Exam  & 35\%.
\end{tabular}
\end{center}
Also note that this is a curved class.\\


\section*{Resources for Students with Disabilities}
The University of Washington is committed to providing access, equal opportunity and reasonable accommodation in its services, programs, activities, education and employment for individuals with disabilities. To request disability accommodation contact the Disability Services Office at: 206-543-8924 or \href{mailto:uwdrs@uw.edu}{uwdrs@uw.edu}.

\section*{Religious Accommodations}
Washington state law requires that UW develop a policy for accommodation of student absences or significant hardship due to reasons of faith or conscience, or for organized religious activities. The UW’s policy, including more information about how to request an accommodation, is available at Religious Accommodations Policy\\ ({https://registrar.washington.edu/staffandfaculty/religious-accommodations-policy/).\\
 Accommodations must be requested within the first two weeks of this course using the Religious Accommodations Request form\\ ({https://registrar.washington.edu/students/religious-accommodations-request/).


\end{document}
