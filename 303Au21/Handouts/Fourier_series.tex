\documentclass[11pt]{article}

\title{Applications of Fouirier series}
\author{Nikolas Eptaminitakis}

\usepackage[T1]{fontenc}
\usepackage{librefranklin}
\renewcommand*\familydefault{\sfdefault} %% Only if the base font of the document is to be sans serif



% Margins
 \setlength{\oddsidemargin}{-.20 in}
 \setlength{\evensidemargin}{-.20 in}
 \setlength{\textwidth}{6.8 in}
 \setlength{\topmargin}{-0.5 in}
 \setlength{\textheight}{9.2 in}
\setlength{\headheight}{0.18 in}

% Packages  
%\usepackage[margin=0.8 in]{geometry} % decides size of margins, paper etc
%\usepackage{palatino} % a font
\usepackage{hhline}
%\usepackage{forest}
\usepackage{cancel} % to cross out stuff
\usepackage{amsmath}
\usepackage{marginnote}  % to put notes on the margins
\usepackage{amssymb}
\usepackage{amsthm}
\usepackage{accents}
\usepackage{amsfonts}
\usepackage{mathtools}
\usepackage{enumerate}
%\usepackage{verbatim}
\usepackage{comment}
%\usepackage{makeidx}
\usepackage{hyperref}
\usepackage{multirow}
\usepackage[noadjust]{cite}
\usepackage{color}
\usepackage{pst-node}
\usepackage{tikz-cd} 
\usepackage[toc,page]{appendix}			% table of contents


% Global options
\allowdisplaybreaks
\mathtoolsset{showonlyrefs}


% Math Operators
\DeclareMathOperator{\sgn}{sgn}
\DeclareMathOperator{\pv}{pv}
\DeclareMathOperator{\Div}{div}
\DeclareMathOperator{\Int}{int}
\DeclareMathOperator{\dist}{dist}
\DeclareMathOperator{\sech}{sech}
\DeclareMathOperator{\sing}{sing}
\DeclareMathOperator{\supp}{supp}
\DeclareMathOperator{\sign}{sign}
\DeclareMathOperator{\II}{II}
\DeclareMathOperator{\diag}{diag}
\DeclareMathOperator{\trace}{tr}
\DeclareMathOperator{\grad}{grad}
\DeclareMathOperator{\adj}{adj}
\DeclareMathOperator{\Span}{span}
\DeclareMathOperator{\Sym}{Sym}
\DeclareMathOperator{\arcsinh}{arsinh}
\DeclareMathOperator{\artanh}{artanh}
\DeclareMathOperator{\proj}{proj}
\DeclareMathOperator{\arcosh}{arcosh}
\DeclareMathOperator{\Diff}{Diff}

%Special Characters
\let \sectionsymbol \S
\renewcommand{\P}{\mathbb{P}}
\newcommand{\R}{\mathbb{R}}
\newcommand{\B}{\mathbb{B}}
\newcommand{\C}{\mathbb{C}}
\renewcommand{\S}{\mathbb{S}}
\renewcommand{\H}{\mathbb{H}}
\newcommand{\Z}{\mathbb{Z}}
\newcommand{\pr}{\mathcal{p}\mathcal{r}}
\newcommand{\ru}{{\sqrt{u}}}
\newcommand{\SM}{\overline{S^*M}}
\newcommand{\pdo}{\Psi\text{DO}}
\newcommand{\ox}{\o{\xi}}
\newcommand{\tM}{\tilde{M}}
\newcommand{\tg}{\tilde{g}}
\newcommand{\p}{\partial}
\newcommand{\n}{\nabla}
\newcommand{\on}{\overline{\n}}
\newcommand{\tn}{\tilde{\nabla}}
\newcommand{\oM}{\overline{M}}
\newcommand{\tX}{\widetilde{X}}
\newcommand{\ta}{{\widetilde{\alpha}}}
\newcommand{\tx}{{\widetilde{x}}}
\newcommand{\oX}{\overline{X}}
\newcommand{\X}{\mathfrak{X}}
\newcommand{\oW}{\overline{W}}
\newcommand{\oT}{\overline{T}}
\newcommand{\tf}{\tilde{f}}
\newcommand{\tr}{\tilde{\r}}
\newcommand{\x}{{x_c}}
\newcommand{\oU}{\overline{U}}
\newcommand{\oH}{\overline{H}}
\newcommand{\oSU}{\o{S^*U}}
\newcommand{\oSH}{\overline{S^*\mathbb{H}^2}}
\newcommand{\rg}{\rangle}
\renewcommand{\lg}{\langle}
\newcommand{\bTM}{{}^bT^*\oM}
\newcommand{\Di}{\Delta\iota}
\newcommand{\hx}{\hat{\xi}}
\newcommand{\ba}{\breve{a}}
\newcommand{\tz}{\widetilde{\zeta}}
\newcommand{\hY}{\hat{Y}}
\newcommand{\tPhi}{\widetilde{\Phi}}
\newcommand{\og}{\overline{\g}}
\newcommand{\prl}{\parallel}
\newcommand{\cM}{\mathring{M}}
\newcommand{\xsp}{x_{\mathrm{sp}}}


% Calligraphic letters

\newcommand{\calA}{\mathcal{A}}
\newcommand{\calB}{\mathcal{B}}
\newcommand{\calC}{\mathcal{C}}
\newcommand{\calD}{\mathcal{D}}
\newcommand{\calE}{\mathcal{E}}
\newcommand{\calF}{\mathcal{F}}
\newcommand{\calG}{\mathcal{G}}
\newcommand{\calH}{\mathcal{H}}
\newcommand{\calL}{\mathcal{L}}
\newcommand{\calO}{\mathcal{O}}
\newcommand{\calR}{\mathcal{R}}
\newcommand{\calS}{\mathcal{S}}
\newcommand{\calU}{\mathcal{U}}
\newcommand{\calV}{\mathcal{V}}
\newcommand{\calW}{\mathcal{W}}
\newcommand{\calY}{\mathcal{Y}}
\newcommand{\calZ}{\mathcal{Z}}


%Greek Letters

\renewcommand{\a}{\alpha}
\renewcommand{\b}{\beta}
\newcommand{\g}{\gamma}
\renewcommand{\d}{\delta}
\let\epsilon\varepsilon
\newcommand{\e}{\epsilon}
\newcommand{\h}{\eta}
\newcommand{\z}{\zeta}
\newcommand{\smsec}{\G_0^{\frac{1}{2}}}
\newcommand{\G}{\Gamma}
\newcommand{\oG}{\overline{\Gamma}}
\renewcommand{\r}{\rho}
\renewcommand{\t}{\tau}
\renewcommand{\k}{\kappa}
\renewcommand{\l}{\lambda}
\newcommand{\s}{\sigma}
\renewcommand{\th}{\theta}
\newcommand{\om}{\omega}
\newcommand{\w}{\omega}
\renewcommand{\oe}{\overline{\eta}}
\newcommand{\tU}{\tilde{U}}

% Environments

\newtheorem{definition}{Definition}
\newtheorem{lemma}{{Lemma}}
\newtheorem{theorem}{Theorem}
\newtheorem{proposition}{Proposition}
\newtheorem{conjecture}{Conjecture}
\newtheorem{remark}{Remark}
\newtheorem{corollary}{Corollary}
\newtheorem{example}{Example}
\newtheorem{ansatz}{Ansatz}
\newtheorem{problem}{Problem}
\newtheorem{question}{Question}
\newenvironment{solution}{\paragraph{Solution:}}{\hfill$\square$}
\newtheorem{goal}{Goal}
\newtheorem{claim}{Claim}
\newenvironment{answer}{\paragraph{Answer:}}{\hfill$\square$}


\let \o \undefined
\def \o#1{\overline{#1}}
\def\fr#1#2{\frac{#1}{#2}}
\def\tt#1{\textit{#1}}

% \let\printintex\undefined
% \let\see\undefined

\let\td\undefined
\def \td#1{\widetilde{#1}}
\let\implies\Rightarrow

\begin{document}
\section*{Applications of Fourier series}

In this worksheet we will use Fourier series to explore the response of spring-mass systems (damped or undamped) to periodic external forces.

\subsection*{Undamped motion}
The equation for the displacement $x$ from equilibrium of an undamped spring-mass system under the influence of an external force $F(t)$ has the form
\begin{equation}
    mx''+kx=F(t),\label{spring-mass}
\end{equation}
where $m$ denotes the mass and $k$ the spring constant. Write $\w_0=\sqrt{k/m}$ for the \textit{natural frequency}  of the system.
The general solution is of the form 
\begin{equation}
    s(t)=A\cos(\w _0 t)+B\sin(\w _0 t)+x_p(t),
\end{equation}
where $x_p(t)$ is a particular solution which depends on the force $F(t)$. Note that the first two terms are independent of the extenal force: they only depend on the parameters of the system and the initial conditions.

Assume that $F(t)$ is piecewise smooth and periodic with period $2L$ \textbf{and that 
$\w_0\neq \pi n/L$ for every integer $n.$} Then we can write a Fourier series for $F$ and use it to find a particular solution of \eqref{spring-mass} which is periodic of period $2L$, in the form
\begin{equation}
    \xsp(t) =\frac{a_0}{2}+\sum_{n=1}^\infty\left( a_n \cos\left(\frac{n\pi t}{L}\right)+b_n \sin\left(\frac{n\pi t}{L}\right)\right). \label{fourier}
\end{equation}
We call the formal solution obtained this way a \textit{steady periodic solution}. 

% We will find a periodic particular solution of \eqref{spring-mass} which we call a  and denote by $x_{\mathrm{sp}}(t)$: 
    

\begin{problem}\label{prob1}
    Suppose that $m=1$, $k=4$, and that the external force $F(t)$ is given by the even function of period $2L=4$ such that $F(t)=2t$ if $0<t<2$.
    \begin{enumerate}
        \item Sketch the graph of $F(t)$ for $-6<t<6$.
        \item Compute the Fourier series of $F(t)$\label{item2}
        \item Plug into \eqref{spring-mass} the Fourier series you found in Part \ref{item2} and the Fourier series for the steady periodic solution in \eqref{fourier} to determine the coefficients $a_n$ and $b_n$.
        In this way you can find the Fourier series expansion of the steady periodic solution.
        \item Write the general solution of \eqref{spring-mass} with for the specific parameters and $F$ given in this problem.
    \end{enumerate}
\end{problem}


\begin{remark}
    As you found in Problem \ref{prob1}, if the  equation has the form in \eqref{spring-mass} (with no term of the form $cx'$ in the left hand side), then if the Fourier series of $F$ has no sine terms, the same holds for the Fourier series \eqref{fourier} of $\xsp$. Similarly, if the Fourier series of $F$ has no cosine terms, the same holds for the one of $\xsp$.
\end{remark}


\subsection*{Pure resonance}
We assumed before that the period of the external force was such that $\w_0\neq n\pi/L$ for all positive integers $n$. If there is any positive integer for which $\w_0= n\pi/L$ then we have the phenomenon of \textit{pure resonance}, which means that the amplitude of the general solution $x(t)$ increases unboundedly.

% To see what goes wrong with the previous method in this case, consider a modified version of Problem \ref{prob1}:
\begin{problem}
    Let $m=1$, $k=\pi^2/4$ in \eqref{spring-mass}, with $F$ as in Problem \ref{prob1}.
    As you already found out, if there is a periodic solution with period $2L=4$ with Fourier series expansion \eqref{fourier}, it will not have sine terms. 
    Assuming that such a Fourier series exists, try to determine $a_0$, $a_1$ and $a_2$. What goes wrong when you try to find $a_1$?
\end{problem}

If the Fourier series of $F$ has the expansion
\begin{equation}\label{fouriersine}
    F(t)= \sum_{n=1}^\infty b_n \sin\left(\frac{n\pi}{L}t\right)
\end{equation}and there is a positive integer $N$ for which $ \frac{N\pi}{L}=\w_0=\sqrt{k/m}$ then the solution  $x(t)$ analogous to \eqref{fourier} is of the form
\begin{equation}
    x(t)=-\frac{b_N}{2m\w_0}t\cos(\w_0 t)+\sum_{n\neq N}\dfrac{b_n}{m(\w_0^2-n^2\pi^2/L^2)}\sin\left(\frac{n\pi t}{L}\right).\label{solution}
\end{equation}
Notice that the first term increases unboundedly and the sine term with the ``problematic'' frequency $N\pi/L$ is not included in the right hand side.
\footnote{To see how the solution \eqref{solution} was derived, you can think of using undetermined coefficients to solve infinitely many differential equations of the form $mx''+kx=b_n\sin(n\pi t/L)$ and superimposing them. For $n=N$, your solution will look like the first term in \eqref{solution}.}
Eq. \eqref{solution} is \textit{not} a Fourier series.

\begin{problem}
    Determine whether pure resonance occurs for the following combinations of parameters $m$, $k$ and $F(t)$:
    \begin{enumerate}
        \item $m=1$, $k=4\pi^2$ and $F(t)$ is the odd function of period 2 with $F(t)=2t$ for $0<t<1$
        \item $m=2$, $k=10$; $F(t)$ is the odd function of period 2 with $F(t)=1$ for $0<t<1$.
    \end{enumerate}
    Bonus: if pure resonance occurs, write a solution in the form \eqref{solution} and plot the first few terms using a computer algebra system.
\end{problem}

\subsection*{Damped forced motion}
In the presence of damping, the equation of motion for the spring mass system under the influence of a periodic external force $F(t)$ becomes
\begin{equation}
    mx''+cx'+kx=F(t),\quad c>0.\label{eq:damped}
\end{equation}
The solution to \eqref{eq:damped} consists of a sum of a steady periodic solution and a transient solution, which decays exponentially fast. If $F(t)$ has an expansion of the form \eqref{fouriersine},
then the steady periodic solution has the expansion
\begin{equation}
    \xsp(t)=\sum_{n=1}^\infty\frac{b_n\sin(\w_nt-\a_n)}{\sqrt{(k-m\w_n^2)^2+(c\w_n)^2}},
\end{equation}
where $\w_n=n\pi/L$ and the phase angle $\a_n$ is the angle satisfying
\begin{equation}
    \tan \a_n=\frac{c\w_n}{k-m\w_n^2} \text{ and }0<\alpha_n<\pi.
\end{equation}

\begin{problem}
    If $m=2$, $c=0.01$, $k=4$, and $F(t)$  is the force in Problem 3 p.2, determine the coefficients and phase angles for the first three nonzero terms of the series corresponding to $\xsp$.
\end{problem}


\end{document}