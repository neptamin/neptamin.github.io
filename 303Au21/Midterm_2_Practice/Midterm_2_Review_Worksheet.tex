\documentclass[11 pt]{article}

\title{}
\author{Nikolas Eptaminitakis}

\usepackage[T1]{fontenc}
\usepackage{librefranklin}
\renewcommand*\familydefault{\sfdefault} %% Only if the base font of the document is to be sans serif



% Margins
 \setlength{\oddsidemargin}{0 in}
 \setlength{\evensidemargin}{0 in}
 \setlength{\textwidth}{6.6 in}
 \setlength{\topmargin}{0 in}
 \setlength{\textheight}{8.55 in}
\setlength{\headheight}{0.18 in}

% Packages  
%\usepackage[margin=0.8 in]{geometry} % decides size of margins, paper etc
%\usepackage{palatino} % a font
\usepackage{hhline}
%\usepackage{forest}
\usepackage{cancel} % to cross out stuff
\usepackage{amsmath}
\usepackage{marginnote}  % to put notes on the margins
\usepackage{amssymb}
\usepackage{amsthm}
\usepackage{accents}
\usepackage{amsfonts}
\usepackage{mathtools}
\usepackage{enumerate}
%\usepackage{verbatim}
\usepackage{comment}
%\usepackage{makeidx}
\usepackage{hyperref}
\usepackage{multirow}
\usepackage[noadjust]{cite}
\usepackage{color}
\usepackage{pst-node}
\usepackage{tikz-cd} 
\usepackage[toc,page]{appendix}			% table of contents


% Global options
\allowdisplaybreaks
\mathtoolsset{showonlyrefs}


% Math Operators
\DeclareMathOperator{\sgn}{sgn}
\DeclareMathOperator{\pv}{pv}
\DeclareMathOperator{\Div}{div}
\DeclareMathOperator{\Int}{int}
\DeclareMathOperator{\dist}{dist}
\DeclareMathOperator{\sech}{sech}
\DeclareMathOperator{\sing}{sing}
\DeclareMathOperator{\supp}{supp}
\DeclareMathOperator{\sign}{sign}
\DeclareMathOperator{\II}{II}
\DeclareMathOperator{\diag}{diag}
\DeclareMathOperator{\trace}{tr}
\DeclareMathOperator{\grad}{grad}
\DeclareMathOperator{\adj}{adj}
\DeclareMathOperator{\Span}{span}
\DeclareMathOperator{\Sym}{Sym}
\DeclareMathOperator{\arcsinh}{arsinh}
\DeclareMathOperator{\artanh}{artanh}
\DeclareMathOperator{\proj}{proj}
\DeclareMathOperator{\arcosh}{arcosh}
\DeclareMathOperator{\Diff}{Diff}

%Special Characters
\let \sectionsymbol \S
\renewcommand{\P}{\mathbb{P}}
\newcommand{\R}{\mathbb{R}}
\newcommand{\B}{\mathbb{B}}
\newcommand{\C}{\mathbb{C}}
\renewcommand{\S}{\mathbb{S}}
\renewcommand{\H}{\mathbb{H}}
\newcommand{\Z}{\mathbb{Z}}
\newcommand{\pr}{\mathcal{p}\mathcal{r}}
\newcommand{\ru}{{\sqrt{u}}}
\newcommand{\SM}{\overline{S^*M}}
\newcommand{\pdo}{\Psi\text{DO}}
\newcommand{\ox}{\o{\xi}}
\newcommand{\tM}{\tilde{M}}
\newcommand{\tg}{\tilde{g}}
\newcommand{\p}{\partial}
\newcommand{\n}{\nabla}
\newcommand{\on}{\overline{\n}}
\newcommand{\tn}{\tilde{\nabla}}
\newcommand{\oM}{\overline{M}}
\newcommand{\tX}{\widetilde{X}}
\newcommand{\ta}{{\widetilde{\alpha}}}
\newcommand{\tx}{{\widetilde{x}}}
\newcommand{\oX}{\overline{X}}
\newcommand{\X}{\mathfrak{X}}
\newcommand{\oW}{\overline{W}}
\newcommand{\oT}{\overline{T}}
\newcommand{\tf}{\tilde{f}}
\newcommand{\tr}{\tilde{\r}}
\newcommand{\x}{{x_c}}
\newcommand{\oU}{\overline{U}}
\newcommand{\oH}{\overline{H}}
\newcommand{\oSU}{\o{S^*U}}
\newcommand{\oSH}{\overline{S^*\mathbb{H}^2}}
\newcommand{\rg}{\rangle}
\renewcommand{\lg}{\langle}
\newcommand{\bTM}{{}^bT^*\oM}
\newcommand{\Di}{\Delta\iota}
\newcommand{\hx}{\hat{\xi}}
\newcommand{\ba}{\breve{a}}
\newcommand{\tz}{\widetilde{\zeta}}
\newcommand{\hY}{\hat{Y}}
\newcommand{\tPhi}{\widetilde{\Phi}}
\newcommand{\og}{\overline{\g}}
\newcommand{\prl}{\parallel}
\newcommand{\cM}{\mathring{M}}


% Calligraphic letters

\newcommand{\calA}{\mathcal{A}}
\newcommand{\calB}{\mathcal{B}}
\newcommand{\calC}{\mathcal{C}}
\newcommand{\calD}{\mathcal{D}}
\newcommand{\calE}{\mathcal{E}}
\newcommand{\calF}{\mathcal{F}}
\newcommand{\calG}{\mathcal{G}}
\newcommand{\calH}{\mathcal{H}}
\newcommand{\calL}{\mathcal{L}}
\newcommand{\calO}{\mathcal{O}}
\newcommand{\calR}{\mathcal{R}}
\newcommand{\calS}{\mathcal{S}}
\newcommand{\calU}{\mathcal{U}}
\newcommand{\calV}{\mathcal{V}}
\newcommand{\calW}{\mathcal{W}}
\newcommand{\calY}{\mathcal{Y}}
\newcommand{\calZ}{\mathcal{Z}}


%Greek Letters

\renewcommand{\a}{\alpha}
\renewcommand{\b}{\beta}
\newcommand{\g}{\gamma}
\renewcommand{\d}{\delta}
\let\epsilon\varepsilon
\newcommand{\e}{\epsilon}
\newcommand{\h}{\eta}
\newcommand{\z}{\zeta}
\newcommand{\smsec}{\G_0^{\frac{1}{2}}}
\newcommand{\G}{\Gamma}
\newcommand{\oG}{\overline{\Gamma}}
\renewcommand{\r}{\rho}
\renewcommand{\t}{\tau}
\renewcommand{\k}{\kappa}
\renewcommand{\l}{\lambda}
\newcommand{\s}{\sigma}
\renewcommand{\th}{\theta}
\newcommand{\om}{\omega}
\newcommand{\w}{\omega}
\renewcommand{\oe}{\overline{\eta}}
\newcommand{\tU}{\tilde{U}}


\newcommand{\NI}{\noindent}


% Environments

\newtheorem{definition}{Definition}
\newtheorem{lemma}{{Lemma}}
\newtheorem{theorem}{Theorem}
\newtheorem{proposition}{Proposition}
\newtheorem{conjecture}{Conjecture}
\newtheorem{remark}{Remark}
\newtheorem{corollary}{Corollary}
\newtheorem{example}{Example}
\newtheorem{ansatz}{Ansatz}
\newtheorem{problem}{Problem}
\newtheorem{question}{Question}
\newenvironment{solution}{\paragraph{Solution:}}{\hfill$\square$}
\newtheorem{goal}{Goal}
\newtheorem{claim}{Claim}
\newenvironment{answer}{\paragraph{Answer:}}{\hfill$\square$}


\let \o \undefined
\def \o#1{\overline{#1}}
\def\fr#1#2{\frac{#1}{#2}}
\def\tt#1{\textit{#1}}

% \let\printintex\undefined
% \let\see\undefined

\let\td\undefined
\def \td#1{\widetilde{#1}}
\let\implies\Rightarrow

\begin{document}
\section*{MA 30300 \\Midterm 2 Review Worksheet}

\begin{itemize}
    \item  Sections covered: 7.1-7.6, 9.1-9.2. Part of 9.3 might be inlcluded. In this worksheet Problems \ref{first}-\ref{last} are from 9.3.



 \item  The Laplace transform table as it appears in p. 781 of the textbook will be provided.


 % \item  A (R) indicates a routine exercise, a (H) indicates a more demanding one.
\end{itemize}



\begin{enumerate}

\item  Compute the Laplace transform $F(s)=\calL\{f(t)\}$, where $f(t)=\sinh(t)$ from the definition, i.e. without using a table (recall that $\sinh(t)=\frac{1}{2}(e^t-e^{-t})$).
For what $s$ is $F(s)$ defined?


\item  Use the Laplace Transform to solve the initial value problem
\begin{equation}
    \begin{cases}
        x'=2x+y\\
        y'=6x+3y
    \end{cases}
\end{equation}
under the initial conditions $x(0)=-1$, $y(0)=2$.


% 7.2 problem 11







    \item  Find the solution to the following initial value problem
    \begin{equation}
        \begin{cases}
          x'''+4x''+4x'=e^{-2t}\\
          x(0)=0,\quad x'(0)=0, \quad x''(0)=0.  
        \end{cases}
        % 
    \end{equation}





\item You are given the following two functions defined for $t\geq 0$:
\begin{equation}
    f(t)=\begin{cases}
        1, \quad 0\leq t\leq\pi\\
        0, \quad \text{otherwise}
    \end{cases}, \quad g(t)=\cos(t).
\end{equation}

Sketch their graphs and compute their convolution $f*g(t)$ for $t\geq 0$. Sketch the graph of the convolution as well.



\item  You are given the following functions defined for $t\geq 0 $
\begin{equation}
    f_\a(t)=\cos(\a t),\quad g(t)=\cos( t),
\end{equation}
where $\a\geq 0$ is a parameter.  
\begin{enumerate}
    \item Compute the convolution $f_\a*g(t)$ for $t\geq 0$, for all values of the parameter $\a\geq  0$.
    \item For what values of $\a$ is $f_\a*g(t)$ bounded as a function of $t$?
    \item For what values of $\a$ is $f_\a*g(t)$ a periodic function of $t$? \\
Hint: When a function is periodic, any integer multiple of a period is also a period. If $\b>0$, what are the periods of the function $\sin(\b t)$?
\end{enumerate}


\item Compute the Laplace transform of the following functions, defined for $t\geq 0$:
\begin{enumerate}
     \item $\displaystyle f(t)=\frac{e^t-e^{-t}}{t}$ 

     \item $g(t)=t^2\cos(2t)$
     \item $h(t)=t^3$ if $1\leq t\leq 2$, $h(t)=0$ otherwise.
 \end{enumerate} 
 %  7.5 22

\item Compute the inverse Laplace transform of $F(s)=\arctan \left(\frac{3}{s+2}\right)$. \\
Hint: First find $\calL^{-1}\{F'(s)\}$.



\item Use properties of the Laplace Transform and the formulas
 $$\calL^{-1}\Big\{\frac{k}{s^2+k^2}\Big\}=\sin(kt),\quad  \calL^{-1}\Big\{\frac{s}{s^2+k^2}\Big\}=\cos(kt)$$ to compute $ \calL^{-1}\Big\{ \frac{1}{(s^2+k^2)^2}\Big\}$.\\
 \textit{Note: The formula for $ \calL^{-1}\Big\{ \frac{1}{(s^2+k^2)^2}\Big\}$ is actually included in the Laplace transform table, but this exercise is asking you to derive it.}



\item Solve the integrodifferential equation describing the current $i(t)$ in an RLC circuit given an impressed voltage $e(t):$
\begin{equation}
    L\frac{di}{dt}+R i+\frac{1}{C}\int_0^t i(\t)d\t =e(t), \quad i(0)=0,
\end{equation}
where
\begin{equation}
    L=1,\quad R=150,\quad C=2\times 10^{-4}, \quad  e(t)=\begin{cases}
        100t, \quad & 0\leq t<1\\
        0, \quad & t\geq 1
    \end{cases}.
\end{equation}



\item Solve the initial value problem
\begin{equation}
    x''+2x'+x=\delta(t)-\delta(t-2), \quad x(0)=x'(0)=2.
\end{equation}


\item Find the weight function (unit impulse response)  for the spring-mass system
\begin{equation}
    mx''+cx'+kx=f(t), \quad x(0)=x'(0)=0,
\end{equation}
where $m=1, $ $c=6$ and $k=9$,
and apply Duhamel's principle to write an integral formula for the solution in terms of the input $f$.

\item Which of the following functions are periodic on $\R$?
\begin{enumerate}
    \item $f_1(t)=\tan(t)$ (assume it is defined to be $0$ for the values of $t$ where $\tan(t)$ is undefined)
    \item $f_2(t)=\sinh(2t)$
    \item $f_3(t)=t\sin(2t)$
    \item $f_4(t)=\arctan(t+1)$
    \item $f_5(t)=\sin(\pi t)+\sin(t)$
\end{enumerate}



\item For each of the functions below defiened on $\mathbb{R}$, decide whether it is odd, even, or neither even nor odd.
\begin{enumerate}[(a)]
    \item $f_1(t)=t^2\cos(t)$
    \item $f_2(t)=t\cos(t)$
    \item $f_3(t)=(t+1)\sin(t)$
    \item $f_4(t)=t^2$
    \item $f_5(t)=\begin{cases}
        t^2,\quad t\geq 0\\
        -t^2,\quad t<0
    \end{cases}$
\end{enumerate}




\item \label{first} Compute the Fourier series for the following periodic functions (assume that their value at points of discontinuity is defined to be the average of their side limits there):
\begin{enumerate}
    \item $\displaystyle f_1(t)=\begin{cases}
        0, & -2<t<0\\t^2, &0<t<2
    \end{cases}$, periodic with period $4$. % 9.2 10
    \item $\displaystyle f_2(t)=t^2, 0<t<2
    $, periodic with period $2$. 
    \item $f_3(t)=\begin{cases}
        0, & -1<t<0
        \\
        \sin(\pi t), &0<t<1
    \end{cases}$, periodic with period $2$. % 13
    

\end{enumerate}
    \NI Which of the functions above, if any, are even? Which ones are odd? For which ones is the term-by-term differentiation of the Fourier series valid?




% \item Use one of cases above to show that 


\item For the following functions defined on intervals of the form $I=[0,L]$, sketch the graphs of their even and odd $2L-$periodic extensions.
Then compute their Fourier sine and cosine series of the original functions (equivalently, the usual Fourier series of the even and odd periodic extensions, respectively):
\begin{enumerate}
    \item $f_1(t)=\cos(t)$ on $I=[0,\pi]$

     \item $f_2(t)=\cos(t)$ on $I=[0,3\pi/2]$
\end{enumerate}


\item\label{last} Find a formal solution for the endpoint problem $x''-4x=1$, $\quad x(0)=x(\pi)=0$

\end{enumerate}

\end{document}