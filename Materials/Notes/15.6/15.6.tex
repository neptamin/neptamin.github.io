\documentclass[12pt]{article}

\usepackage[margin=0.8 in]{geometry}
\usepackage{amsmath}
\usepackage{amssymb}
\usepackage{macros}
\usepackage{mathtools}
\usepackage{enumerate}
\usepackage{verbatim}
\usepackage{amsthm}

\title{}
%\content{}



\let \proj \undefined
\renewcommand{\tr}{ \mathrm{tr}}
\DeclareMathOperator{\SU}{SU}
\DeclareMathOperator{\proj}{proj}
\newcommand{\sS}{\mathscr{S}}
\DeclareMathOperator{\comp}{comp}
\newcommand{\A}{\mathcal{A}}
\renewcommand{\D}{\mathcal{D}}
\renewcommand{\e}{\epsilon}
\newcommand{\Are}{\A_{r,\e}}
\newcommand{\Kre}{K_{r,\e}}
\newcommand{\Dre}{\D_{r,\e}}
\newcommand{\rt}{\tilde{r}}
\newcommand{\et}{\tilde{\e}}
\newtheorem{definition}{Definition}
\newenvironment{solution}
  {\begin{proof}[Solution]}
  {\end{proof}}
\newtheorem{example}{Example}
\newtheorem{exercise}{Exercise}

\newcommand{\vr}{\mathbf{r}}
\newcommand{\vF}{\mathbf{F}}

\newtheorem{theorem}{Theorem}



\begin{document}
\begin{example}
Find the surface area of the sphere centered at the origin with radius $r$.
\end{example}
\begin{solution}
As a remark, the fact that the sphere is centered at the origin doesn't matter, as its surface area doesn't depend on the center! The equation of the sphere is $$x^2+y^2+z^2=R^2.$$ Note that we've seen how to calculate the surface areas of graphs of functions, but the sphere is \textbf{not} the graph of a function (why?). However, both its upper and lower hemisphere \textbf{are} graphs of functions and, in fact, they have the same surface area. So we'll consider the function \begin{equation}\label{upper}z=f(x,y)=\sqrt{R^2-x^2-y^2}\end{equation}  representing the upper hemisphere. To find the domain of integration, we intersect with the $xy$ plane and find, by \eqref{upper}, $$z=0\implies x^2+y^2=R^2.$$ So, the domain of integration has to be the disk $$D=\{(x,y):x^2+y^2\leq R^2\}.$$

Then we compute partial derivatives and find $$f_x(x,y)=\frac{-2x}{2\sqrt{R^2-x^2-y^2}}$$ and $$f_y(x,y)=\frac{-2y}{2\sqrt{R^2-x^2-y^2}}.$$ Therefore, \begin{align*}
S=&\iint_D\sqrt{(f_x(x,y))^2+(f_y(x,y))^2+1}dA\\
=&\iint_D\frac{R}{\sqrt{R^2-x^2-y^2}}dA\\
=&\int_0^{2\pi}\int_0^R\frac{Rr}{\sqrt{R^2-r^2}}drd\theta\\
=&\int_0^{2\pi}\int_{R}^0-\frac{R}{2}u^{-\frac{1}{2}}dud\theta\\
=&2\pi R^2.
\end{align*}

Recall that all this is only for the upper hemisphere, so to find the total surface area we multiply $\times 2 $ and we find $$Area=4\pi R^2.$$
\end{solution}



\end{document}

