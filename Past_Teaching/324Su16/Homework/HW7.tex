

\documentclass[10pt,letter]{article}
	% basic article document class
	% use percent signs to make comments to yourself -- they will not show up.
\usepackage{enumerate}
\usepackage{amsmath}
\usepackage{amssymb}
	% packages that allow mathematical formatting

\usepackage{graphicx}
	% package that allows you to include graphics

\usepackage{setspace}
	% package that allows you to change spacing

\onehalfspacing
	% text become 1.5 spaced

\usepackage{fullpage}
	% package that specifies normal margins
	\DeclareMathOperator{\curl}{curl}
		\DeclareMathOperator{\grad}{grad}
		\DeclareMathOperator{\dive}{div}

\newcommand{\F}{\mathbf{F}}
\renewcommand{\i}{\mathbf{i}}
\renewcommand{\j}{\mathbf{j}}
\renewcommand{\k}{\mathbf{k}}
\newcommand{\sint}{\iint_S\curl\F\cdot d\mathbf{S}}
\newcommand{\lint}{\int_C\F\cdot d\mathbf{r}}
\begin{document}
	% line of code telling latex that your document is beginning


\title{Homework Set 7}

%\author{Your Name Here}

\date{Not due}
	% Note: when you omit this command, the current dateis automatically included
 
\maketitle 
	% tells latex to follow your header (e.g., title, author) commands.

\section*{Section 16.8}
\paragraph{Exercise:} Use Stokes' Theorem to evaluate $\sint$:\begin{enumerate}
\item $\F(x,y,z)=x^2z^2\i+y^2z^2\j+xyz\k$, $S$ is the part of the paraboloid $z=x^2+y^2$ that lies inside the cylinder $x^2+y^2=25$, oriented upward. (answer: 0)
\item $\F(x,y,z)=xyz\i+xy\j+x^2yz\k$, $S$ consists of the top and four sides (but not the bottom) of the cube with vertices ($\pm 10,\pm 10,\pm 10$), oriented outward. (answer: 0)
\end{enumerate}
%$\F(x,y,z)=xyz\i+xy\j+x^2yz\k$

\paragraph{10:} Use Stokes' Theorem to evaluate $\lint$, where $C$ is oriented counterclockwise as viewed from above: $\F(x,y,z)=xy\i+4z\j+7y\k$, $C$ is the line of intersection of the plane $x+z=8$ and the cylinder $x^2+y^2=9$. (answer: 27$\pi$)

\paragraph{17:}
A particle moves along line segments from the origin to the points (3, 0, 0), (3, 5, 1), (0, 5, 1), and back to the origin under the influence of the force field $\F(x,y,z)=z^2\i+3xy\j+3y^2\k$. Find the work done. (answer: 219/2).


\section*{Section 16.9}
\paragraph{5:} Use the Divergence Theorem to find the surface integral $\iint_S\F\cdot d\mathbf{S}$, $\F(x,y,z)=xye^z\i+xy^2z^3\j-ye^z\k$, $S$ is the surface of the box bounded by the coordinate planes and the planes $x=7$, $y=8$ and $z=1$. (answer: 392)
\paragraph{Exercise:} Verify that the Divergence Theorem is true for the vector field $\F$ on the region $E$: $\F(x,y,z)=x^2\i+xy\j+z\k$, $E$ is the solid bounded by the paraboloid $z=9-x^2-y^2$ and the $xy$-plane. (answer: $\frac{81}{2}\pi)$
\paragraph{9:} Use the Divergence Theorem to calculate the surface integral $\iint_S\F\cdot d\mathbf{S}$, where $\F(x,y,z)=x^2\sin y\i+x\cos y\j-xz\sin y \k$, $S$ is the "fat sphere" $x^8+y^8+z^8=125$. (Answer: 0)
\end{document}
