

\documentclass[10pt,letter]{article}
	% basic article document class
	% use percent signs to make comments to yourself -- they will not show up.
\usepackage{enumerate}
\usepackage{amsmath}
\usepackage{amssymb}
	% packages that allow mathematical formatting

\usepackage{graphicx}
	% package that allows you to include graphics

\usepackage{setspace}
	% package that allows you to change spacing

\onehalfspacing
	% text become 1.5 spaced

\usepackage{fullpage}
	% package that specifies normal margins
	\DeclareMathOperator{\curl}{curl}
		\DeclareMathOperator{\grad}{grad}
		\DeclareMathOperator{\dive}{div}

\newcommand{\F}{\mathbf{F}}
\renewcommand{\i}{\mathbf{i}}
\renewcommand{\j}{\mathbf{j}}
\renewcommand{\k}{\mathbf{k}}
\newcommand{\sint}{\iint_S\F\cdot d\mathbf{S}}

\begin{document}
	% line of code telling latex that your document is beginning


\title{Homework Set 6}

%\author{Your Name Here}

\date{Due: Friday August 12 }
	% Note: when you omit this command, the current dateis automatically included
 
\maketitle 
	% tells latex to follow your header (e.g., title, author) commands.

\paragraph{}
This assignment contains a lot of problems of parametrizing surfaces. Once you've found a parametrization, you can use Wolfram Mathematica or other similar software to plot your equations and see if they give you the correct picture.
\paragraph{Turn in the following problems:}
\section*{16.6}
%\paragraph{2:} Determine whether the points $P(3, 3, 2)$ and $Q(4, −2, −10)$ lie on the surface $\mathbf{r}(u, v)=\langle u + v, u^2-v, u + v^2\rangle$.

\paragraph{19: } Find a parametrization for the plane through the origin that contains the vectors $ \i-\k$ and $\j-\k$.
\paragraph{24:} Find a parametrization for the part of the sphere $x^2+y^2+z^2=9$ that lies between the planes $z=-2$ and $z=2$.
\paragraph{25:} Find a parametric representation for the part of the cylinder $y^2+z^2=121$ that lies between the planes $x=0$ and $x=3$.
\paragraph{29:} Find parametric equations for the surface obtained by rotating the curve $y=e^{-x}$, $0\leq x\leq 3$ around the $x-$axis.
\paragraph{37:} Find an equation of the tangent plane to the given parametric surface $\mathbf{r}(u, v) = u^2 \i + 6u \sin v \j + u \cos v \k$ at the point $u = 2$ , $v = 0$.
\paragraph{41:} Find the area of the part of the plane $x + 2y + 3z = 1$ that lies inside the cylinder $x^2 + y^2 = 6$.(answer: $2\sqrt{14}\pi$)
\paragraph{45:} Find the area of the part of the surface $z = xy$ that lies within the cylinder $x^2 + y^2 = 9$. (answer: $\frac{2}{3}(10\sqrt{10}-1)\pi$)

\section*{16.7:}
\paragraph{17:} Evaluate the surface integral $\iint_S(x^2z+y^2z)dS$, where $S$ is the hemisphere $x^2+y^2+z^2=4$, $z\geq 0$. (Answer: 16$\pi$).
\paragraph{Exercise:} Evaluate the surface integral $\iint_SxydS$, where $S$ is the triangular region with vertices (1,0,0), (0,6,0), (0,0,6). (answer: $3\sqrt{19/2}$)
\paragraph{Exercise II:} Evaluate the surface integral $\sint$ for the given vector fields $\F$ and the oriented surface $S$. In other words, find the flux of $\F$ across $S$. For closed surfaces use the positive (outward) orientation.

\begin{enumerate}
\item $\F(x,y,z)=x\i+y\j+z^4\k$, $S$ is the part of the cone $z=\sqrt{x^2+y^2}$ below the plane $z=1$ with downward orientation. (answer: $\pi/3$)
\item $\F(x,y,z)= x\i-z\j+y\k$ and $S$ is the part of the sphere $x^2+y^2+z^2=36$ in the first octant, with orientation toward the origin. (answer: $-36\pi$)
\item $\F(x,y,z)= xz\i+x\j+y\k$, $S$ is the hemisphere $x^2+y^2+z^2=9$, $y\geq 0$, oriented in the direction of the positive y-axis. (answer: 0)
\item $\F(x,y,z)= x\i+2y\j+3z\k$, $S$ is the cube with vertices $(\pm 1, \pm1,\pm1)$. (answer: 48)
\item $\F(x,y,z)= x\i+y\j+9\k$, where $S$ is the boundary of the region enclosed by the cylinder $x^2+z^2=1$ and the planes $y=0$ and $x+y=8$ (answer: 16$\pi$).
\end{enumerate}





\end{document}
