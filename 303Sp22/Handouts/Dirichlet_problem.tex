\documentclass[11pt]{article}

\title{}
\author{Nikolas Eptaminitakis}

\usepackage[T1]{fontenc}
\usepackage{palatino}
% \renewcommand*\familydefault{\sfdefault} %% Only if the base font of the document is to be sans serif



% Margins
 \setlength{\oddsidemargin}{0 in}
 \setlength{\evensidemargin}{0 in}
 \setlength{\textwidth}{6.6 in}
 \setlength{\topmargin}{0 in}
 \setlength{\textheight}{8.55 in}
\setlength{\headheight}{0.18 in}

% Packages  
%\usepackage[margin=0.8 in]{geometry} % decides size of margins, paper etc
%\usepackage{palatino} % a font
\usepackage{hhline}
%\usepackage{forest}
\usepackage{cancel} % to cross out stuff
\usepackage{amsmath}
\usepackage{marginnote}  % to put notes on the margins
\usepackage{amssymb}
\usepackage{amsthm}
\usepackage{accents}
\usepackage{amsfonts}
\usepackage{mathtools}
\usepackage{enumerate}
%\usepackage{verbatim}
\usepackage{comment}
%\usepackage{makeidx}
\usepackage{hyperref}
\usepackage{multirow}
\usepackage[noadjust]{cite}
\usepackage{color}
\usepackage{pst-node}
\usepackage{tikz-cd} 
\usepackage[toc,page]{appendix}			% table of contents


% Global options
\allowdisplaybreaks
\mathtoolsset{showonlyrefs}


% Math Operators
\DeclareMathOperator{\sgn}{sgn}
\DeclareMathOperator{\pv}{pv}
\DeclareMathOperator{\Div}{div}
\DeclareMathOperator{\Int}{int}
\DeclareMathOperator{\dist}{dist}
\DeclareMathOperator{\sech}{sech}
\DeclareMathOperator{\sing}{sing}
\DeclareMathOperator{\supp}{supp}
\DeclareMathOperator{\sign}{sign}
\DeclareMathOperator{\II}{II}
\DeclareMathOperator{\diag}{diag}
\DeclareMathOperator{\trace}{tr}
\DeclareMathOperator{\grad}{grad}
\DeclareMathOperator{\adj}{adj}
\DeclareMathOperator{\Span}{span}
\DeclareMathOperator{\Sym}{Sym}
\DeclareMathOperator{\arcsinh}{arsinh}
\DeclareMathOperator{\artanh}{artanh}
\DeclareMathOperator{\proj}{proj}
\DeclareMathOperator{\arcosh}{arcosh}
\DeclareMathOperator{\Diff}{Diff}

%Special Characters
\let \sectionsymbol \S
\renewcommand{\P}{\mathbb{P}}
\newcommand{\R}{\mathbb{R}}
\newcommand{\B}{\mathbb{B}}
\newcommand{\C}{\mathbb{C}}
\renewcommand{\S}{\mathbb{S}}
\renewcommand{\H}{\mathbb{H}}
\newcommand{\Z}{\mathbb{Z}}
\newcommand{\pr}{\mathcal{p}\mathcal{r}}
\newcommand{\ru}{{\sqrt{u}}}
\newcommand{\SM}{\overline{S^*M}}
\newcommand{\pdo}{\Psi\text{DO}}
\newcommand{\ox}{\o{\xi}}
\newcommand{\tM}{\tilde{M}}
\newcommand{\tg}{\tilde{g}}
\newcommand{\p}{\partial}
\newcommand{\n}{\nabla}
\newcommand{\on}{\overline{\n}}
\newcommand{\tn}{\tilde{\nabla}}
\newcommand{\oM}{\overline{M}}
\newcommand{\tX}{\widetilde{X}}
\newcommand{\ta}{{\widetilde{\alpha}}}
\newcommand{\tx}{{\widetilde{x}}}
\newcommand{\oX}{\overline{X}}
\newcommand{\X}{\mathfrak{X}}
\newcommand{\oW}{\overline{W}}
\newcommand{\oT}{\overline{T}}
\newcommand{\tf}{\tilde{f}}
\newcommand{\tr}{\tilde{\r}}
\newcommand{\x}{{x_c}}
\newcommand{\oU}{\overline{U}}
\newcommand{\oH}{\overline{H}}
\newcommand{\oSU}{\o{S^*U}}
\newcommand{\oSH}{\overline{S^*\mathbb{H}^2}}
\newcommand{\rg}{\rangle}
\renewcommand{\lg}{\langle}
\newcommand{\bTM}{{}^bT^*\oM}
\newcommand{\Di}{\Delta\iota}
\newcommand{\hx}{\hat{\xi}}
\newcommand{\ba}{\breve{a}}
\newcommand{\tz}{\widetilde{\zeta}}
\newcommand{\hY}{\hat{Y}}
\newcommand{\tPhi}{\widetilde{\Phi}}
\newcommand{\og}{\overline{\g}}
\newcommand{\prl}{\parallel}
\newcommand{\cM}{\mathring{M}}


% Calligraphic letters

\newcommand{\calA}{\mathcal{A}}
\newcommand{\calB}{\mathcal{B}}
\newcommand{\calC}{\mathcal{C}}
\newcommand{\calD}{\mathcal{D}}
\newcommand{\calE}{\mathcal{E}}
\newcommand{\calF}{\mathcal{F}}
\newcommand{\calG}{\mathcal{G}}
\newcommand{\calH}{\mathcal{H}}
\newcommand{\calL}{\mathcal{L}}
\newcommand{\calO}{\mathcal{O}}
\newcommand{\calR}{\mathcal{R}}
\newcommand{\calS}{\mathcal{S}}
\newcommand{\calU}{\mathcal{U}}
\newcommand{\calV}{\mathcal{V}}
\newcommand{\calW}{\mathcal{W}}
\newcommand{\calY}{\mathcal{Y}}
\newcommand{\calZ}{\mathcal{Z}}


%Greek Letters

\renewcommand{\a}{\alpha}
\renewcommand{\b}{\beta}
\newcommand{\g}{\gamma}
\renewcommand{\d}{\delta}
\let\epsilon\varepsilon
\newcommand{\e}{\epsilon}
\newcommand{\h}{\eta}
\newcommand{\z}{\zeta}
\newcommand{\smsec}{\G_0^{\frac{1}{2}}}
\newcommand{\G}{\Gamma}
\newcommand{\oG}{\overline{\Gamma}}
\renewcommand{\r}{\rho}
\renewcommand{\t}{\tau}
\renewcommand{\k}{\kappa}
\renewcommand{\l}{\lambda}
\newcommand{\s}{\sigma}
\renewcommand{\th}{\theta}
\newcommand{\om}{\omega}
\newcommand{\w}{\omega}
\renewcommand{\oe}{\overline{\eta}}
\newcommand{\tU}{\tilde{U}}

% Environments

\newtheorem{definition}{Definition}
\newtheorem{lemma}{{Lemma}}
\newtheorem{theorem}{Theorem}
\newtheorem{proposition}{Proposition}
\newtheorem{conjecture}{Conjecture}
\newtheorem{remark}{Remark}
\newtheorem{corollary}{Corollary}
\newtheorem{example}{Example}
\newtheorem{ansatz}{Ansatz}
\newtheorem{problem}{Problem}
\newtheorem{question}{Question}
\newenvironment{solution}{\paragraph{Solution:}}{\hfill$\square$}
\newtheorem{goal}{Goal}
\newtheorem{claim}{Claim}
\newenvironment{answer}{\paragraph{Answer:}}{\hfill$\square$}


\let \o \undefined
\def \o#1{\overline{#1}}
\def\fr#1#2{\frac{#1}{#2}}
\def\tt#1{\textit{#1}}

% \let\printintex\undefined
% \let\see\undefined

\let\td\undefined
\def \td#1{\widetilde{#1}}
\let\implies\Rightarrow

\begin{document}

\noindent
\Large
\textbf
{MA 30300\\
The Dirichlet problem in polar domains}

\medskip

\normalsize


\noindent We learn how to use separation of variables to solve the Dirichlet problem on a domains $R$ in the plane which can be expressed easily in polar coordinates $(r,\th)$ with $x=r \cos(\th)$ and $y=r\sin(\th)$.
Let $R$ be of the form 
\begin{equation}
    R=\{(r,\theta):\alpha <r<\b , 0\leq \theta <2\pi \}.
\end{equation}
Here we will work with the case where $\a>0$, $\b<\infty$ (annulus). In the textbook you can find a solved example with $\a=0$ (disk\footnote{Technically the domain becomes a punctured disk, but by assuming that your solution is continuous at the origin you are actually solving the Dirichlet problem in a disk.}), and in the homework you will solve the Dirichlet problem in the exterior of a disk, so with $\a>0$ and $\b=\infty$.
In polar coordinates $(r,\theta)$, the Laplacian takes the form
\begin{equation}
    \nabla^2u=u_{rr}+\frac{1}{r}u_r+\frac{1}{r^2}u_{\th\th}.
\end{equation}

\smallskip

Start with the following warm-up problems:

\begin{problem}[Warm-up]\label{prob:1}
Let $\l$ be a real number. 
We would like to find a non-trivial function $\Theta(\theta)$ satisfying the following:
% \begin{equation}
    \begin{align}
        \Theta''(\th)+\l \Theta(\th)=0& \text{ for } \theta\in \R\label{eq:ode}\\
        \Theta(\th)=\Theta(\th+2\pi).&\label{period}
    \end{align}
% \end{equation}
For what values of $\l$ is this possible? What are the corresponding solutions?\\
Hint: consider separately the cases $\l<0$, $\l=0 $, $\l>0$ and find the corresponding general solutions of \eqref{eq:ode} and determine for what $\l$ eq. \eqref{period} can be true.
\end{problem}

\begin{problem}[Warm-up]
    Find the general solution $R(r)$ to the following problem:
    \begin{equation}
        r^2 R''+r R'=0.\label{eq:prob1}
    \end{equation}
    Hint: set $u=R'$ and notice that the resulting differential equation is separable.
\end{problem}

\begin{problem}[Warm-up]\label{prob:3}
    Now look at the following problem:
    \begin{equation}
        r^2 R''+r R'-n^2 R=0,\label{eq:prob2}
    \end{equation}
    where $n$ is a positive integer.
    This is a second order linear equation with non-constant coefficients; we have not seen a general method for solving such an equation. 
    If you can find two linearly independent solutions $R_1$ and $R_2$, any other solution can be written as $R=AR_1+BR_2$.

    Find two linearly independent solutions in the following way: make the educated guess that the solutions will be of the form $R=r^k$, plug into the equation and find the $k$ for which the equation is satisfied.
\end{problem}

Let $R=\{(r,\theta):\alpha <r<\b , 0\leq \theta <2\pi \}$ be an annulus as before, $\a>0$.
Now we would like to find $u(r,\th)$ which is $2\pi$ periodic in $\theta$ and solves the problem 
% \begin{equation}
     \begin{align}
         u_{rr}+\frac{1}{r}u_r+\frac{1}{r^2}u_{\th\th}=0&\quad \text{ on }R\label{eq:pde}\\*
         u(\a,\th)=f(\th),&\quad  0\leq \th< 2\pi\label{eq:a}\\*
         u(\b,\th)=0, &\quad 0\leq \th< 2\pi\label{eq:b}
     \end{align}
 % \end{equation} 
 We seek a solution in the form $$u(r,\th)=\sum_{n=0}^\infty c_n u_n(r,\th)$$ as usual.
 We would like to ensure that $u_n$ satisfy $\eqref{eq:pde}$ and \eqref{eq:b}.
 We will determine $c_n$ at the end so that the sum satisfies \eqref{eq:a} as well.
 
 \label{a}Follow the steps below to find such a solution:
 \begin{enumerate}
     \item Assume that $u_n(r,\th)=R_n(r)\Theta_n(\th)$. Plug $u_n$ into the PDE \eqref{eq:pde} and separate variables (all functions of $r$ on the left, all functions of $\th$ on the right).
     \item Deduce that $\Theta_n $ satisfies an ODE of the form \eqref{eq:ode}.
     Since we want $u(r,\theta)=u(r,\theta+2\pi)$, we would like the same to be true for $u_n$.
     Conclude that $\Theta_n$ also satisfies \eqref{period}.

     \item Use the result of Problem \ref{prob:1} to deduce that $\l=0$ or $\l=n^2$, for $n$ a positive integer.

     \item Deduce that $R_n$ must satisfy either \eqref{eq:prob1} or \eqref{eq:prob2}.
     Determine the form of $R_n$ so that $u_n$ also satisfies \eqref{eq:b}.

     \item Use the conclusions of Problems \ref{prob:1}-\ref{prob:3} to deduce that
     \begin{equation}\label{eq:sum}
         u(r,\theta)=A_0\ln \left(\frac{r}{\b}\right)+\sum_{n=1}^\infty \left( (r/\b)^n-(r/\b)^{-n}\right)(A_n\cos(n\th)+B_n\sin(n\th))
     \end{equation}

     \item Now plug in $r=\a$ into \eqref{eq:sum}. You want \eqref{eq:a} to be true.
     Recall that $f(\theta)$ is a periodic function of period $2\pi$.
     What should the coefficients $A_0$, $A_n$ and $B_n$ be?\\
     Hint: recall that $f$ has a Fourier series expansion.
 \end{enumerate}




\end{document}